%! Author = smen851
%! Date = 11/07/2024

% Preamble

\documentclass[11pt, a4paper]{scrartcl}

% Packages


\usepackage[utf8]{inputenc}

\usepackage{pgfplots}
\usepackage{amsmath}
% lorem package is used to generate random text for testing
\usepackage{lipsum}
\usepackage{bold-extra}
\usepackage{geometry}
\usepackage{import}
\usepackage{graphicx}
\usepackage{layouts}
\usepackage{caption}
\captionsetup[figure]{name={Fig.}}


\geometry{a4paper, total={170mm,257mm}, left=25mm, right=25mm, top=10mm, bottom=25mm}
%%Macros
\graphicspath{ {../src/figures/} }
\newcommand{\sect}[1]{Section~\ref{#1}}
\newcommand{\degrees}{\ensuremath{^\circ}}
\renewcommand{\figurename}{Fig.}
% workaround for pgfplots
\def\mathdefault#1{#1}


% Title
\title{Dynamic Antenna Pattern and SAR Performances: \\Notes on Methods and Implementation}
\author{\texttt{Simone Mencarelli}}
\date{\small{\texttt{July 2024}}}
% Document
\begin{document}
    \maketitle

    textwidth in pt: \the\textwidth \\
    textwidth in cm: \printinunitsof{cm}\prntlen{\textwidth}\\
    textwidth in inch: \printinunitsof{in}\prntlen{\textwidth}\\


    \section{Antenna Model}
    \label{sec:antenna_model}
    Assuming the SAR antenna is a \emph{ReflectArray}~(RA), e.g., like the JPL radar~\cite{swot}.
    The conformal surface (of the deployable structure) is modeled as an array of \emph{atomic} re-readiating elements.
    For the purpose of this model the implementation of the atomic element is not relevant, and we will be concerned only
    with the surface impedance properties of each cell.
    Summarizing~\cite{MetaTutorialLiu2023}, with a variable local reactive surface impedance, the reflection pattern can
    be shaped and steered in directions different from the specular one.
    This property of RA (and generally of meta-surfaces) is called \emph{anomalous reflection} and can be used, for example, to approximate the reflection properties
    of a parabolic reflector using a flat surface~\cite{rapozar}; or to shape the far field pattern~\cite{Guarrielllo} of a reflector antenna.
    In practice, the surface impedance modulation, is obtained varying some physical parameters of the atomic element
    like a printed patch size, or the distance and thickness of a series of printed concentric rings or squared rings;
    or again, any structure exhibiting some sort of resonant behavior.
    Examples in literature relevant to flown or hypothetical space missions include elements of the square patch
    type~\cite{swot,marco,isara}, or phoenix cells~\cite{Guarrielllo,secondphoenix}.
    The latter exibiting a \emph{phoenix rebirth} property, meaning that the shape of the elements is similar for
    relative small reflection phase--shift differences also at the 2$\pi$--0 transition in a circular fashion.
    Moreover, phoenix cells offer more degrees of freedom, possibly allowing for better optimization opportunities.
    Other RA cells with a high number of degrees of freedom are possible exploiting complex geometries or multi-layer structures
    to improve specific aspects like the cross--polarization reflection coefficient, e.g.~\cite{PradoCrosspolar2017}.
    RA are, in general, designed under the so-called \emph{Local Periodicity Approximation} (LPA), i.e., the individual
    atoms reflection properties are characterised in the specular reflection direction assuming a periodic surface made
    of identical elements; it follows that the spatial sampling of the surface has to be smaller or equal to half a
    wavelength ($\lambda/2$) to avoid multimodal reflections~\cite{MetaTutorialLiu2023}.
    When modulating (and truncating) the surface impedance of the reflector, in practice, the periodicity is broken and multiple
    reflection mechanisms are allowed~\cite{Esposti2022,NayeriRAbook}.
    It is difficult, however, to predict the exact multi-modal reflection pattern unless a full-wave simulation is
    performed which might be very time and memory--consuming for such electrically large structures.
    For this reason, the classical design approach involves designing the RA cells under the LPA and then validating the
    design with a full-wave simulation or measurements~\cite{NayeriRAbook,rapozar,PradoDatabase2022}.
    To control unwanted reflections, the reflector is illuminated with some taper towards the edges to reduce the contribution
    of edge diffraction and the aperiodicity caused by fast-varying impedance profiles~\cite{NayeriRAbook}.
    To understand the effects of mechanical deformations on the antenna pattern, only the anomalous reflection
    is considered assuming an ideal RA with lossless phase--shifting elements.
    Furthermore, especially for a large reflector with small f/D ratios, the anomalous reflection will be
    particularly sensitive to shape variations of the surface due to the relative added phase-delay.
    Also, it makes sense to consider a uniform illumination of the reflecting surface to account for the worst-case scenario
    and fully characterise the effects of any deformation.
    The physical-optics model to compute the (anomalous reflection component of the) antenna pattern is described in~\sect{subsec:physical_optics_conformal_reflectarray_model}.
    The deformation model and the resulting patterns are described in~\sect{subsec:deformation_model}.
    Some considerations on the symmetries of the antenna pattern, to reduce the computation time, are given in~\sect{subsec:symmetries}.

    \subsection{Physical-Optics Conformal Reflectarray Model}
    \label{subsec:physical_optics_conformal_reflectarray_model}
    The integration procedure is the same of~\cite{PradoCrosspolar2017}.\\
    In short:\\
    \noindent\rule{\textwidth}{0.4pt}
    \begin{ttfamily}
        \small
        Assuming every cell is a $\lambda/2$ square uniform aperture:
        \begin{enumerate}
            \item Compute Tangential E-field illumination.
            \item \textbf{For} every cell:
            \begin{enumerate}
                \item Compute $\mathbf{E}_{r}^{x,y} = \mathbf{\sigma} \mathbf{E}_{i}^{x,y}$.
                \item Compute Reflected field $\mathbf{E}_{r}$-$\mathbf{H}_{r}$ (elements LCS):\\
                $\mathbf{E}_{r} \cdot \mathbf{k} = 0$ ; $\mathbf{H}_{r} = \frac{(\mathbf{k} \times \mathbf{E}_{r})}{\eta}$.
                \item Compute element pattern assuming magnetic and electric current \\excitations:
                ${E}_{r}^{x,y}$ and ${H}_{r}^{x,y}$.
            \end{enumerate}
            \item Conformal Array summation (elemental pattern rotation, displacement and sum).
        \end{enumerate}
    \end{ttfamily}
    * to be expanded *\\
    \noindent\rule{\textwidth}{0.4pt}


    \begin{figure}[!t]
        \includegraphics[width=\textwidth]{rageometry}
        \caption{Reflectarray geometry and required phase shift for broadside ($\mathbf{\hat{z}}$) collimation.
        The feed is suspended 1 m above the reflector center; the overall size for the reflector is 2~m in
        length and 0.3~m in width, the surface is cylindrical with cross--section corresponding to a 90\degrees
        subtended angle.}
        \label{fig:rageom}
    \end{figure}

    \begin{figure}[!bt]
        \centering
        \import{./figures}{linear_pattern.pgf}
        \caption{Far-field pattern for uniformly illuminated reflectarray in the nominal state.
            Elevation coordinate limited to~$\theta \in [0,20]$\degrees
        }
        \label{fig:pattern}
    \end{figure}

    \subsection{Deformation And Dynamic Antenna Pattern}
    \label{subsec:deformation_model}
    The deformation vector is retrieved from the mechanical modal analysis and scaled according to the deformation state
    identified by the oscillation phase $\varphi$ and the maximum amplitude.
    Cell origin deformation vector is computed as the average of the deformation vectors of the four corners of the cell.
    The position of each reflectarray cell is then shifted by summing the averaged deformation vector and rotated
    according to an ortho-normal basis, obtained by averaging and Gram-Schmidt orthogonalization of the deformed mesh
    cell edge vectors.
    The far field is recomputed as in~\sect{subsec:physical_optics_conformal_reflectarray_model} with the updated cell
    positions, keeping the same surface impedance profile $\mathbf{\sigma}$ required for collimation in the nominal state.
    \begin{figure}[!bt]
        \centering
        \import{./figures}{deform_pattern.pgf}
        \caption{Far-field pattern for uniformly illuminated reflectarray in the extreme deformed state, i.e.,
            amplitude scaling~=~1, oscillation phase~=~90\degrees.
        Elevation coordinate limited to~$\theta \in [0,20]$\degrees}
        \label{fig:deformpattern}
    \end{figure}

    \subsection{Symmetries}
    \label{subsec:symmetries}


    \section{SAR Model}
    \label{sec:sar_model}

    \subsection{Geometrical and Stationary Phase Transformations}
    \label{subsec:transformations}

    \subsection{Ambiguity, SNR, and Pd with Dynamic Antenna Pattern}
    \label{subsec:ambiguity_snr_pd_with_dynamic_antenna_pattern}

    \subsection{Unambiguous Mode Design Point}
    \label{subsec:unambiguous_mode_design_point}

    \subsection{Ambiguous Mode Design Point}
    \label{subsec:ambiguous_mode_design_point}

%% Bibliography
    \bibliographystyle{ieeetr}
    \bibliography{../src/main.bib}


\end{document}
