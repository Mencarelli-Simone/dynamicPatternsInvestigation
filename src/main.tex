%! Author = smen851
%! Date = 11/07/2024

% Preamble

\documentclass[11pt, a4paper]{scrartcl}

% Packages
\usepackage{amsmath}
% lorem package is used to generate random text for testing
\usepackage{lipsum}
\usepackage{geometry}
\geometry{a4paper, total={170mm,257mm}, left=25mm, right=25mm, top=10mm, bottom=25mm}
%%Bibliography


% Title
\title{Dynamic Antenna Pattern and SAR Performances: \\Notes on Methods and Implementation}
\author{\texttt{Simone Mencarelli}}
\date{\small{\texttt{July 2024}}}
% Document
\begin{document}
    \maketitle


    \section{Antenna Model}
    \label{sec:antenna_model}
    Assuming the SAR antenna is a \emph{ReflectArray}~(RA), e.g., like the JPL radar~\cite{swot}.
    The conformal surface (of the deployable structure) is modeled as an array of \emph{atomic} re-readiating elements.
    For the purpose of this model the implementation of the atomic element is not relevant, and we will be concerned only
    with the surface impedance properties of each cell.
    Summarizing~\cite{MetaTutorialLiu2023}, with a variable local reactive surface impedance, the reflection pattern can
    be shaped and steered in directions different from the specular one.
    This property of RA (and generally of meta-surfaces) is called \emph{anomalous reflection} and can be used, for example, to approximate the reflection properties
    of a parabolic reflector using a flat surface~\cite{rapozar}; or to shape the far field pattern~\cite{Guarrielllo} of a reflector antenna.
    In practice, the surface impedance modulation, is obtained varying some physical parameters of the atomic element
    like a printed patch size, or the distance and thickness of a series of printed concentric rings or squared rings;
    or again, any structure exhibiting some sort of resonant behavior.
    Examples in literature relevant to flown or hypothetical space missions include elements of the square patch
    type~\cite{swot,marco,isara}, or phoenix cells~\cite{Guarrielllo,secondphoenix}.
    The latter exibiting a \emph{phoenix rebirth} property, meaning that the shape of the elements is similar for
    relative small reflection phase--shift differences also at the 2$\pi$--0 transition in a circular fashion.

    \subsection{Physical-Optics Conformal Reflectarray Model}
    \label{subsec:physical_optics_conformal_reflectarray_model}
    pippo~\cite{PradoCrosspolar2017}.

    \subsection{Deformation And Dynamic Antenna Pattern}
    \label{subsec:deformation_model}

    \subsection{Symmetries}
    \label{subsec:symmetries}


    \section{SAR Model}
    \label{sec:sar_model}

    \subsection{Geometrical and Stationary Phase Transformations}
    \label{subsec:transformations}

    \subsection{Ambiguity, SNR, and Pd with Dynamic Antenna Pattern}
    \label{subsec:ambiguity_snr_pd_with_dynamic_antenna_pattern}

    \subsection{Unambiguous Mode Design Point}
    \label{subsec:unambiguous_mode_design_point}

    \subsection{Ambiguous Mode Design Point}
    \label{subsec:ambiguous_mode_design_point}

    %% Bibliography
    \bibliographystyle{ieeetr}
    \bibliography{../src/main.bib}


\end{document}
